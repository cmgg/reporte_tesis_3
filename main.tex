\documentclass{article}

\usepackage[spanish]{babel}
\usepackage[utf8]{inputenc}
\usepackage{csquotes}
\usepackage{biblatex}
\usepackage{graphicx}
\usepackage{lipsum}

\addbibresource{bibliography.bib}
\graphicspath{ {./Figuras/} }

\title{Reporte de Tesis 3}
\author{ 
Carlos Manuel Gutiérrez Galán \and
Carlos Soubervielle (Asesor) \and
Marco Tulio (Comité) \and
Cesar Guerra (Comité)
}
\date{Junio 2022}

\begin{document}

\begin{titlepage}
    \begin{center}
        {\huge \bfseries Reporte de tesis III} \\[6.5ex]
        {\large \bfseries Carlos Gutiérrez} \\
        \vspace{3ex}
        {\large Carlos Soubervielle (asesor)} \\
        \vspace{3ex}
        {\large Marco Tulio (sinodal técnico)\hspace{1em}Cesar Guerra (sinodal monitor)} \\
        \vspace{3ex}
        Julio del 2022 \\
        \vspace{2ex}
        {\textit{``Design of a Low-Power Gateway to Connect a Weather Monitoring WSN to the Internet through Wireless Protocols''}}
    \end{center}
    \vspace{1ex}
    \begin{abstract}
        \lipsum[1-2]
    \end{abstract}
\end{titlepage}

\tableofcontents\thispagestyle{empty}\newpage\addtocontents{toc}{\protect\thispagestyle{empty}}\setcounter{page}{1}

\section{Problema a resolver}

\subsection{Resumen}
\paragraph{}
Una de las maneras de coordinar y almacenar la información generada por una red de sensores inalámbricos (WSN, por sus siglas en inglés) Zigbee, cuyo propósito es el monitoréo de condiciones meteorológicas, es mediante un módulo\linebreak ``XStick'' y una computadora. El uso de una computadora no solo aumenta\linebreak considerablemente el costo del proyecto si no que también descarta por completo la idea de tener la WSN desconectada de la red eléctrica debido al alto consumo energético. La otra manera es mediante un Gateway propietario de una empresa tercera que usualmente son de caja cerrada y cuyos protocolos inalámbricos son limitados, además de que tienen un costo elevado y se corre el riesgo de que sean descontinuados o se vuelvan obsoletos. El depender de un controlador que es propietario y de diseño cerrado limita mucho la escalabilidad del proyecto, aumenta considerablemente el costo, y restringe la posibilidad de reducir el consumo energético. Esto último tiene impacto cuando la solución requiere tiempos prolongados a la intemperie y se alimenta mediante baterías.

\paragraph{}
La solución a estos problemas es el desarrollar un Gateway a la medida que pueda reemplazar a dicho controladores, integrando las características relevantes de ellos en un sistema embebido, lo cual nos permitirá reducir costos y consumo de energía, así como también abrirle las puertas a la escalabilidad para que el proyecto pueda ser retomado en un futuro, o bien, pueda ser adaptado para resolver otro tipo de problemas. Este Gateway será desarrollado en una plataforma de desarrollo cuyo núcleo principal es un system on a chip (SoC). El software embebido desarrollado permitirá controlar los diferentes componentes del sistema, así como activarlos y desactivarlos en tiempo real para reducir el\linebreak consumo energético del sistema. La selección de dicha plataforma se llevará a cabo mediante el análisis de requerimientos del problema a resolver, y se considerará la metodología co-design que facilita el desarrollo de sistemas que involucran la integración de hardware y software.

\newpage

\subsection{Explicación}
\paragraph{}
Actualmente existe una gran demanda de dispositivos IoT como lo son sensores, actuadores y cámaras, para ser usados en Smart Cities, vehículos autónomos y la denominada ``Industria 4.0''. Un cuello de botella de estos sistemas es la comunicación de datos hacia Internet para su posterior procesamiento en computadoras que poseen más recursos de cómputo, lo cuál es un reto ya que dichos dispositivos IoT se comunican con una diversa cantidad de protocolos de comunicación como lo son Zigbee, Bluetooth, Thread, Ethernet, entre otros, que requieren de un dispositivo que maneje dichos protocolos para la intercomunicación entre ellos y el Internet; Dicho dispositivo se denomina ``Gateway''.

\paragraph{}
Para este proyecto se cuenta con una WSN que permite monitorear diferentes parámetros meteorológicos. Esta WSN está compuesta por diversos nodos ``End~Device'' (nodos~ED) que monitorean las condiciones meteorológicas con sus diversos sensores, para luego comunicar esta información a nodos ``Router'' (nodos~R) y estos a su vez se comunican con un nodo controlador (nodo~C), el cual manda los datos de la red hacia el internet. La implementación actual de la red utiliza una laptop y un XStick para fungir como controlador, lo cuál no es lo óptimo y definitivamente no es la opción indicada para una versión final del proyecto debido a las siguientes restricciones:

\subparagraph{Consumo energético}
El alto consumo energético de una computadora (de escritorio, laptop, servidor, Raspberry Pi, etc) limita el dominio de ubicaciones en las que la WSN puede operar a aquellas que disponen de la red eléctrica, lo cuál no es deseable cuando se busca monitorear condiciones ambientales (por ejemplo en un bosque, un monte, un desierto, etc). Para superar esta limitación se requiere de un sistema con la menor cantidad de hardware que permita la comunicación de los nodos al Internet, y el único software que se ejecute en el sea el que controla la red cuando sea necesario.

\subparagraph{Escalabilidad}
El uso de dispositivos propietarios cerrados como el ``XStick'' crea la dependencia de una compañía tercera que lo manufactura, y que podría descontinuarlo en cualquier momento. Por otro lado también se depende de ella para actualizaciones de firmware, lo cuál podría exponer vulnerabilidades en el sistema en el caso de que estas no sean parchadas por el propio fabricante. Por ser de arquitectura cerrada, no permite ninguna alteración por parte del desarrollador, lo cuál limita en gran medida las opciones de escalabilidad del sistema. Es por esto que se optó por desarrollar nuestro propio controlador, el cuál sea de arquitectura abierta y cuyo firmware sea de código abierto para evitar estos problemas a futuro. Idealmente el Gateway debe de poseer una gran variedad de interfaces y soportar diversos protocolos de comunicación para adicionar funcionalidades a futuro.
% Datos celulares, control de energía, seguridad y cifrado de la información

\subparagraph{Mantenibilidad}
La mantenibilidad de este proyecto permitirá que sea fácil para otros poder basarse en el para escalarlo o desarrollar proyectos similares. Un diseño intuitivo, en donde los componentes estén claramente separados y documentados permitirá que pueda ser retomado rápidamente y las modificaciones pertinentes puedan ser implementadas sin gran problema.

\subparagraph{Costo} \label{sec:costo}
Para que el proyecto sea factible a gran escala es de vital importancia el reducir su costo. El uso de una computadora y el hardware propietario del XStick, hacen que este se dispare por arriba de los \$300 USD. Por otro lado, la propuesta de hardware actual reduce este costo a poco más de \$75 USD\cite{digikey_xstick} (a la fecha de este reporte).

\subparagraph{Tamaño}
Las dimensiones del proyecto son importantes, principalmente por el tema de las \textit{smart cities}. Un sistema pequeño se puede mezclar fácilmente en entornos como los son edificios y hogares inteligentes.

\subparagraph{Usabilidad}
La usabilidad es uno de los principales factores de este proyecto. Por un lado debe de contar con una interfase sencilla con la cuál el usuario final pueda interactuar, y que le permita controlar la WSN sin requerir de un conocimiento técnico relacionado a su funcionamiento. Por otra parte, el firmware y el software del sistema deben de permitirle al desarrollador escalar el producto fácilmente en caso de requerirse a futuro. Es también de gran importancia que la WSN pueda recibir actualizaciones directamente desde el Gateway, para evitar así la tediosa tarea de cargar una laptop hacia cada nodo de la red y conectarla mediante de un programador para actualizar su firmware.

\newpage

\section{Literatura} % State of the art, 4 artículos, 1,000 palabras
Para el estado del arte se encontraron artículos de tres trabajos similares, cuyos resúmenes se encontrarán a continuación

\subsection{Design, Implementation and Evaluation of an IoT\linebreak Home Automation System\cite{home_auto}}
Este artículo trae a la mesa el tema de \textit{fog IoT computing}, la cuál se encarga de traer los servicios de la nube a nodos IoT. Para la comunicación de dichos nodos a el internet, los autores diseñaron, implementaron y evaluaron un Gateway similar al que se propone en este proyecto de tesis. Dicho Gateway utilizará Zigbee para comunicarse con los nodos de cómputo y WiFi para comunicarlos con el internet. El artículo también menciona que ambas tecnologías coexisten en la banda de 2.4 GHz, lo cuál genera colisiones. El sistema que ellos proponen permite la integración de ambas tecnologías de manera fluida y libre de colisiones.

\subsection{IoT Platform for Real-Time Monitoring\cite{iot_platform}}
Al igual que en el artículo anterior, los autores desarrollan un Gateway IoT para conectar una red de sensores al internet. La principal diferencia es que los autores proponen el uso de u conjunto de hardware específico (una computadora Raspberry Pi adicionada con un módulo propietario de Xbee para darle capacidades de comunicación vía  Zigbee) y no toman en cuenta las colisiones dado que su Gateway utiliza solo un protocolo a la vez.

\subsection{From WSN towards WoT\cite{wsn_to_wot}}
El tercer artículo habla de los diferentes medios de comunicación entre diversas WSN locales, y proponen un estándar para realizar una \textit{Web Of Things (WoT)}, análoga al internet. En el artículo se muestra dicha implementación haciendo uso de la especificación de \textit{OpenAPI}, entre otras herramientas. Dicha especificación permite generalizar bajo un mismo estándar las diferentes interfaces de software que exponen las redes locales de dispositivos IoT.

\newpage

\section{Trabajo previo} % 400 palabras
El objetivo a largo plazo de este proyecto es el de diseñar, implementar y validar una WSN con una capacidad aproximada de 300 nodos que permitan medir y almacenar la información de variables meteorológicas. También se busca el poder interactuar con dicha red de sensores mediante el internet.
\subsection{Tesis de licenciatura en Ingeniería en Computación de Arturo López}
En las primeras etapas del proyecto se realizó un trabajo de tesis de licenciatura titulado "Diseño de una micro-estación meteorológica capaz de generar una red de sensores inalámbrica" presentada por el Ing. Arturo López Rangel. En ese trabajo se implementó el primer prototipo de estación meteorológica, el cuál era capaz de realizar mediciones de radiación solar, temperatura, humedad relativa y presión barométrica. Los resultados obtenidos de las mediciones atmosféricas eran después enviados a una computadora mediante el protocolo de Zigbee, haciendo uso de un XStick (propietario de XBee) para funcionar de intermediario entre la computadora y la WSN. La escala de este trabajo fue una única micro-estación conectada al coordinador (PC y XStick) mediante USB y Zigbee.
\subsection{Tesis de maestría de Ingeniería de la Computación de Luis Ortega}
La siguiente etapa del proyecto fue realizada como parte del trabajo de tesis de maestría (aún no concluido) del Ing. Luis Ortega Gutiérrez, el cuál retoma la red WSN y la escala agregándole el uso de los denominados \textit{componentes virtuales}, los cuales permiten configurar los nodos de la WSN de manera dinámica y remota. Esto funciona a través de archivos de configuración que le permiten al usuario configurar los sensores del nodo. La escala de este proyecto fue de 3 nodos conectados a un coordinador mediante Zigbee. Dicho coordinador sigue siendo una computadora y un XStick, lo cuál se busca reemplazar en el trabajo actual.

\newpage

\section{Propuesta de trabajo} % 2,500 palabras

\subsection{Objetivo general}

El objetivo de este proyecto es el de diseñar, desarrollar y validar un Gateway IoT de bajo consumo de energía, cuya función será la de intermediario (o puente) entre una red inalámbrica de sensores meteorológicos Zigbee y una aplicación alojada en la nube encargada de monitorear y controlar dicha red.

\subsection{Objetivos específicos}

\begin{enumerate}
\item Definir los requerimientos de hardware y software.
\item Seleccionar la plataforma de hardware y software adecuada para el diseño del sistema embebido de acuerdo a los requerimientos propuestos. % Bajo consumo de energía, WiFi, Zigbee, múltiples interfaces seriales
\item Adaptar la metodología co-design de forma tal que facilite el desarrollo y la integración del sistema.
\item Revisar las características y capacidades de la plataforma seleccionada para mapear la distribución óptima de tareas.
\item Diseñar la arquitectura de hardware y software y el flujo de trabajo de los componentes en base a la metodología de co-design, considerando los enfoques de \textit{top-down} y \textit{bottom-up}.
\item Desarrollo y validación del software embebido para comunicar la WSN y el Gateway.
\item Desarrollo y validación del software embebido para conectar el Gateway a Internet.
\item Validar la integración de toda la solución de software embebido.
\item Verificar que la solución satisfaga los requisitos.
\end{enumerate}

\newpage

\subsection{Justificación}
Revisando los trabajos revisados en el estado del arte se puede observar que son pensados para interiores, en dónde disponen de acceso a la red energética por lo que el bajo consumo de energía no fue considerado como requerimiento durante su diseño; Esto limita el dominio de aplicación de dichos sistemas a ciudades y edificios. Además, uno de los artículos propone el uso de computadoras y hardware propietario que pueden elevar considerablemente los costos cuando se busca realizar una implementación de WSNs a gran escala. Para el proyecto de esta tesis se pretende desarrollar un Gateway similar a los ya mencionados, pero cuyas principales diferencias serán las siguientes

\paragraph{Consumo de energía}
El aporte más grande de este proyecto es el de diseñar el firmware y software teniendo como requerimiento el bajo consumo de energía. Esto le permitirá ser utilizado en un rango mayor de aplicaciones, en la cuales no se cuente con una red eléctrica que alimente el sistema. La principal desventaja de buscar el bajo consumo de energía es que se puede llegar a perder de manera parcial (o total) el procesamiento en tiempo real, sin embargo esto no es algo que se requiere para aplicaciones de monitoreo de sensores, en donde no es crítico el envío de información de manera inmediata.

\paragraph{Bajo costo}
De igual manera, para este proyecto se busca el reducir costos considerablemente. Ambos Gateways propuestos en la literatura proponen de el uso de computadoras embebidas, como la Raspberry Pi. Dichas computadoras disponen de una gran cantidad de recursos que no son requeridos para este tipo de aplicaciones, y solamente aumentan el costo. Para este proyecto se busca diseñar un sistema embebido basándose en plataformas de desarrollo existentes (en vez de computadoras), buscando así reducir considerablemente los costos. A cambio de esto, el desarrollador ahora cuenta con menos recursos para desarrollar su firmware, lo cual le impone un reto adicional.

\paragraph{Usabilidad}
Una característica de los artículos es que los usuarios finales de sus sistemas son los mismos autores, quienes tienen el conocimiento técnico necesario para utilizar dichos Gateways. A diferencia de ellos ellos, nosotros buscamos que el proyecto sea fácil de usar, tanto por los usuarios finales como por los desarrolladores. Para lograr esto se desarrollará una GUI que le permita al usuario interactuar fácilmente con el sistema, el cuál a su vez será desarrollado de manera concisa y modular para facilitarle la tarea a futuros desarrolladores que deseen retomarlo.

\paragraph{Mantenibilidad y escalabilidad}
Como se acaba de mencionar, se busca que este proyecto sea fácil de ser retomado por futuros desarrolladores. Para esto se busca que el diseño y la implementación sean mantenibles, en donde las abstracciones de los componentes estén bien definidas y documentadas, lo cual facilite modificaciones posteriores. También se desea que sea fácil de escalar, por lo que se seleccionó una plataforma que soporta diversos protocolos de comunicación, así como también posee un GPIO extenso para la fácil adición de módulos.

\subsection{Alcance}
El alcance de este proyecto se divide en dos campos. por un lado se tiene el Gateway y su comunicación con la WSN, y por el otro la comunicación del Gateway hacia el internet.

\paragraph{Software embebido}
Se diseñará, implementará y validará el Gateway de bajo consumo energético y su software embebido. El Gateway realizará la lectura y almacenamiento de los datos generados por la WSN, permitirá configurar los tiempos de \textit{sleep mode} de los motes (nodo de sensores), y además podrá activar y desactivar las variables que serán medidas por cada uno de ellos. El software embebido le permitirá al Gateway comunicarse mediante WiFi con una aplicación de front-end con la cuál el usuario podrá interactuar con el y controlar dichos parámetros de la WSN.

\paragraph{Software}
La aplicación de software permitirá al usuario controlar el Gateway y la WSN desde un ordenador conectado mediante WiFi a la misma red local que ellos. La aplicación será desarrollada en Python en conjunto con los frameworks de Tkinter para la GUI, Flask para el servidor backend y SQLite para el almacenamiento de los datos. Esta aplicación le permitirá al usuario leer el historial de datos de la WSN, activar o desactivar los sensores de los motes, y cambiar la frecuencia de muestreo de los datos.

% Aportación:
% - Plataforma de desarrollo de sistemas embebidos
% - WiFi y Zigbee
% - Low-power
% - Bajo costo
% - Usabilidad, mantenibilidad y escalabilidad

\subsection{Metodología de trabajo}
% ¿Cómo se implementara la metodología co-design?
% Hacer uso extenso de diagramas UML

\subsection{Avances del proyecto}

\subsubsection{Análisis de los requerimientos} \label{sec:reqs}
El objetivo principal de este proyecto es el de desarrollar un Gateway de bajo consumo de energía que permita al usuario interactuar con la WSN a través del internet. Dado que la WSN utiliza Zigbee para comunicación inalámbrica, el Gateway debe de tener soporte para dicho protocolo, preferentemente ya integrado en el hardware seleccionado facilitar las tareas de diseño y desarrollo. En tema de trabajos futuros, idealmente el hardware poseerá soporte para varios protocolos de comunicación serial, además de poseer una cantidad considerable de pines GPIO, lo cuál permitirá la escalabilidad del proyecto permitiendo adicionarle nuevos módulos, características y funcionalidades. La idea de este Gateway es la de desplegar a gran escala las WSN, por lo que es crítico reducir los costos fijos del hardware. Ya por último, se busca que sea fácil el interactuar con la WSN desde internet, ya que el usuario final no necesariamente será alguien experto en materia, por tanto es importante el desarrollo de una GUI para el front-end del sistema que le permita leer los datos de la WSN y reconfigurarla en tiempo real.


En resumen, los requerimientos son
\begin{itemize}
\item Bajo consumo de energía.
\item Bajo costo.
\item Poseer comunicación mediante Zigbee y WiFi.
\item Soportar múltiples protocolos seriales.
\item Poseer un extenso GPIO.
\item Facilidad de uso por parte del usuario.
\item Facilidad de ser mantenido y escalado en trabajos futuros.
\end{itemize}

\subsubsection{Herramientas de hardware~y~software}
\paragraph{Hardware}
Se optó por utilizar el SoC QCA4020 de la empresa Qualcomm porque cumple con todos los requisitos ya mencionados. Permite modos de bajo consumo de energía en sus tres núcleos, tiene un costo de aproximadamente \$75 USD por lo cuál es bastante accesible en comparación a las soluciones mencionadas en la sección \ref{sec:costo}. Soporta una gran cantidad de protocolos seriales como I\textsuperscript{2}C, UART, HS UART, QSPI, USB2.0, entre otras. Tiene un núcleo específico para la comunicación por Zigbee y otro para WiFi. En la sección \ref{sec:doc_hw} se describirá más a detalle el hardware.

\paragraph{Software embebido}
El software embebido será desarrollado en C, haciendo uso del SDK proporcionado por Qualcomm. Cabe mencionar que, a pesar de existir alternativas como micropython o rust, se eligió C por que permite un mayor control sobre del hardware, lo cuál es vital cuando se busca reducir el consumo de energía.

\paragraph{Software front-end}
Para el desarrollo del front-end se seleccionó Python por su conveniencia y diversidad de módulos, utilizando flask para el servidor y tkiner para la GUI. Para la base de datos se utilizará SQLite por su facilidad de ser integrada con flask.

\subsubsection{Documentación de la plataforma seleccionada} \label{sec:doc_hw}
Se seleccionó el kit de desarrollo para el SoC QCA4020 ya que facilita a gran medida la implementación y depuración del software embebido ya que expone directamente la interfaz JTAG a USB. El SoC QCA4020 cuenta con diversas características que satisfacen todos los requerimientos del proyecto.

\paragraph{Núcleos y consumo de energía}
El SoC posee tres núcleos de procesamiento:
\begin{itemize}
\item Un ARM Cortex-M4F como núcleo de procesamiento de la aplicación.
\item Un ARM Cortex-M0 encargado del stack de comunicación para el estándar IEEE 802.15.4, el cuál controla los stacks de Bluetooth 5, Zigbee y Thread.
\item Un Tensilica Xtensa de 130 MHz, dedicado exclusivamente a WiFi.
\end{itemize}

Estos núcleos poseen características que les permite ser apagados o entrar en modos de bajo consumo de energía, lo cuál es vital para este proyecto.

\paragraph{Interfaces y protocolos seriales}
El SoC posee las siguientes interfaces
\begin{itemize}
\item I\textsuperscript{2}S
\item SDIO2.0
\item SPI/Q-SPI
\item I\textsuperscript{2}C
\item GPIO
\item SPI
\item UART
\end{itemize}

Además tiene 8 interfaces PWM optimizadas para iluminación LED y un ADC de 8 canales, con una resolución de 12 bits, y una tasa de muestreo de 1Msps.

\subsection{Entrenamiento en la plataforma y SDK}
% - Descripción breve del SDK (módulos, librerías, herramientas, etc)
% - Entorno de desarrollo

% \section{Caso de uso del sistema} ------- Pendiente para el siguiente reporte
% Diagramas UML de casos de uso del usuario

\section{Evaluación del trabajo}
El proyecto será evaluado con base a casos de uso que cubran todos los requerimientos de la sección \ref{sec:reqs}

\section{Cronograma}
% Incluir el trabajo planeado para las vacaciones
% Fecha de fin planeada para febrero 2023

\section{Comentarios}
% Escalabilidad, mantenibilidad y modularidad (de HW y de SW)
% Usabilidad a través de una GUI en la app del back-end

\section{Bibliografía}
\printbibliography
% 2 referencias de SoC
% 4 Literatura
% 1 Co-design
% Trabajo previo: tesis de Arturo y artículo de Luis Octavio

\end{document}
